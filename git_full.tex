\documentclass[10pt]{examdesign}
\usepackage{amsmath}
\usepackage{pifont}
\usepackage{textcomp}
\SectionFont{\large\sffamily}
%\ShortKey
% Disable answer key
\NoKey
\Fullpages
\ContinuousNumbering
\DefineAnswerWrapper{}{}
\NumberOfVersions{1}
% Disable the alarming examish stuff at the top
\def\namedata{}


\class{{\Large Git review}}
\examname{Quiz}

\begin{document}

\begin{matching}[title={Basic concepts (1 pt each)}]
  Match the following concepts with their definitions.
  \pair{working tree}{directory containing the files you are currently working on}
  \pair{commit}{a full snapshot of a working tree}
  \pair{repository}{a DAG of commits with additional information (e.g., branches, tags, remotes)}
  \pair{remote}{another repository whose branches your repository tracks}
  \pair{branch}{a pointer to a particular commit, which moves forward as you commit}
  \pair{tag}{a pointer to a particular commit}
  \pair{merge}{joins two or more development histories together}
  \pair{staging area}{where snapshots of changes are placed before they are committed}
  \pair{push and pull}{used to synchronize repositories}
\end{matching}

\begin{shortanswer}[title={Short Answer (2 pts each)},
                    rearrange=no,resetcounter=no]

\begin{block}[questions=3]
Please explain what each of the following terms means.

\begin{question}
  \texttt{HEAD}
  \vspace{5mm}
  \begin{answer}
    \texttt{HEAD} is a pointer to the current commit you are on.
  \end{answer}
\end{question}

\begin{question}
  \texttt{master}
  \vspace{5mm}
  \begin{answer}
    When Git creates a repository, \texttt{master} is the name given
    to the default branch.  It is often used as the main ``trunk'' from
    which branches diverge (possibly to be merged back into the
    trunk later).
  \end{answer}
\end{question}

\begin{question}
  \texttt{origin}
  \vspace{5mm}
  \begin{answer}
    The default name for your main remote.  For instance, if you created
    your repository by cloning it, Git will automatically set \texttt{origin}
    to point to this remote.
  \end{answer}
\end{question}
\end{block}

\begin{block}[questions=3]
Please answer the following questions.

\begin{question}
  Which of the above terms are fixed and which are used by convention?
  \vspace{5mm}
  \begin{answer}
    Both \texttt{master} and \texttt{origin} are default conventions, which
    you may easily change (this is not normally recommended).  While
    \texttt{HEAD} is fixed.
  \end{answer}
\end{question}

\begin{question}
  What \texttt{git} command would you use to see what \texttt{master}
  specifically refers to in one of your repositories?
  \vspace{5mm}
  \begin{answer}
    \texttt{git branch -v}
  \end{answer}
\end{question}

\begin{question}
  What \texttt{git} command would you use to see what \texttt{origin}
  specifically refers to in one of your repositories?
  \vspace{5mm}
  \begin{answer}
    \texttt{git remote -v}
  \end{answer}
\end{question}

\end{block}


\begin{block}[questions=3]
After you execute some \texttt{git} command, \texttt{git} prints
an informative message starting with the phrase:

\vspace{4mm}
\texttt{You are in \textquotesingle detached HEAD\textquotesingle\  state.}

\begin{question}
  What \texttt{git} command did you enter?
  \vspace{5mm}
  \begin{answer}
    \texttt{git checkout <commit hash>}
  \end{answer}
\end{question}

\begin{question}
  What does the phrase
  \texttt{\textquotesingle detached HEAD\textquotesingle}\ mean?
  \vspace{5mm}
  \begin{answer}
    A \texttt{\textquotesingle detached HEAD\textquotesingle}\ occurs
    when checkout a commit that is \textbf{not} a proper, local,
    branch name. In other words, it just means you are no longer
    working on a branch (or tag).
  \end{answer}
\end{question}

\begin{question}
  How would you ``reattach'' \texttt{HEAD}? 
  \vspace{5mm}
  \begin{answer}
    You either need to checkout an existing branch or create a new
    branch where you are.
  \end{answer}
\end{question}

\end{block}

\end{shortanswer}

\begin{shortanswer}[title={Longer Answer (5 pts each)},
                    rearrange=no,resetcounter=no]


\begin{block}[questions=2]
You are collaborating on a project with one of your colleagues. And you
are using Git to version control the project. To integrate your work,
you are using a repository hosting service like GitHub.  Please
answer the following.

\begin{question}
  You've just committed a new working tree to your local branch.  Now
  you want to share your changes with your colleague, so you type
  \texttt{git push} and get the following helpful message:

  \texttt{! [rejected]        master -> master (non-fast-forward) \\
    error: failed to push some refs to \textquotesingle<your shared remote repository>\textquotesingle \\
    hint: Updates were rejected because the tip of your current branch is behind \\
    hint: its remote counterpart. Integrate the remote changes (e.g. \\
    hint: \textquotesingle git pull ...\textquotesingle) before pushing again. \\
    hint: See the \textquotesingle Note about fast-forwards\textquotesingle\ in \textquotesingle git push --help\textquotesingle\ for details.
  }

  Please explain what happened, why it happened, and how to resolve it.
  What does \emph{fast-forward} mean?
  \vspace{80mm}
  \begin{answer}
    Just read and rephrase the error message. From Bash, type \\
    \texttt{\$ git help glossary} \\
    and search for \emph{fast-forward}.
  \end{answer}
\end{question}

\begin{question}
  When you execute \texttt{git pull}, you get a merge conflict.
  So you type \texttt{git status} and see this:

  \texttt{\# On branch master \\
    \# You have unmerged paths. \\
    \#   (fix conflicts and run "git commit") \\
    \# \\
    \# Unmerged paths: \\
    \#   (use "git add ..." to mark resolution) \\
    \# \\
    \# both modified:      README.md \\
    \# \\
    no changes added to commit (use "git add" and/or "git commit -a")
  }

  It appears that you've both modified \texttt{README.md}, so you open
  it in your text editor and see:

  \texttt{The sum is: \\
    <<<<<<< HEAD \\ 
    2 \\
    ======= \\
    3 \\
    >>>>>>> master \\
  }

  Explain (1) what happened; (2) what \texttt{<<<<<<< HEAD}, \texttt{=======},
  and \texttt{>>>>>>> master} mean; and (3) what you would do to resolve
  the conflict (including how you would commit and push the resolution).
  \vspace{50mm}
  \begin{answer}
    The conflict marked area of the text begins with \texttt{<<<<<<< HEAD}
    and ends with \texttt{>>>>>>> master}.  The two conflicting sets of
    changes are separated by \texttt{=======}.  These symbols are the
    conflict markers.

    To resolve the conflict, you could choose one of the changes and delete
    the other along with the conflict markers.  Often the conflicts will be
    more involved and you may desire to integrate the two changes rather
    than choosing one over the other.

    Once you have resolved the conflict in the file that you both modified,
    make sure that you have a valid file (e.g., if you are working on code,
    you need to check that your resolution hasn't created an error).  In
    particular, you must remove the conflict markers as they are unlikely
    to be valid in your programming language (nor are they often desired
    as part of your text).

    After you've verified that your resolution hasn't introduced new errors,
    you will want to add, commit, and push (maybe you need to pull before
    you push).
  \end{answer}
\end{question}

\end{block}


\end{shortanswer}

\end{document}
