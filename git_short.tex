\documentclass[10pt]{examdesign}
\usepackage{amsmath}
\usepackage{pifont}
\usepackage{textcomp}
\SectionFont{\large\sffamily}
%\ShortKey
% Disable answer key
\NoKey
\Fullpages
\ContinuousNumbering
\DefineAnswerWrapper{}{}
\NumberOfVersions{1}
% Disable the alarming examish stuff at the top
\def\namedata{}


\class{{\Large Git review}}
\examname{Quiz}

\begin{document}

\begin{matching}[title={Basic concepts}]
  Match the following concepts with their definitions.
  \pair{working tree}{directory containing the files you are currently working on}
  \pair{commit}{a full snapshot of a working tree}
  \pair{repository}{a DAG of commits with additional information (e.g., branches, tags, remotes)}
  \pair{remote}{another repository whose branches your repository tracks}
  \pair{branch}{a pointer to a particular commit, which moves forward as you commit}
  \pair{tag}{a pointer to a particular commit}
  \pair{merge}{joins two or more development histories together}
  \pair{staging area}{where snapshots of changes are placed before they are committed}
  \pair{push and pull}{used to synchronize repositories}
\end{matching}

\begin{shortanswer}[title={Short Answer},
                    rearrange=no,resetcounter=no]

\begin{block}[questions=3]
Please explain what each of the following terms means.

\begin{question}
  \texttt{HEAD}
  \vspace{5mm}
  \begin{answer}
    \texttt{HEAD} is a pointer to the current commit you are on.
  \end{answer}
\end{question}

\begin{question}
  \texttt{master}
  \vspace{5mm}
  \begin{answer}
    When Git creates a repository, \texttt{master} is the name given
    to the default branch.  It is often used as the main ``trunk'' from
    which branches diverge (possibly to be merged back into the
    trunk later).
  \end{answer}
\end{question}

\begin{question}
  \texttt{origin}
  \vspace{5mm}
  \begin{answer}
    The default name for your main remote.  For instance, if you created
    your repository by cloning it, Git will automatically set \texttt{origin}
    to point to this remote.
  \end{answer}
\end{question}
\end{block}

\begin{block}[questions=3]
Please answer the following questions.

\begin{question}
  Which of the above terms are fixed and which are used by convention?
  \vspace{5mm}
  \begin{answer}
    Both \texttt{master} and \texttt{origin} are default conventions, which
    you may easily change (this is not normally recommended).  While
    \texttt{HEAD} is fixed.
  \end{answer}
\end{question}

\begin{question}
  What \texttt{git} command would you use to see what \texttt{master}
  specifically refers to in one of your repositories?
  \vspace{5mm}
  \begin{answer}
    \texttt{git branch -v}
  \end{answer}
\end{question}

\begin{question}
  What \texttt{git} command would you use to see what \texttt{origin}
  specifically refers to in one of your repositories?
  \vspace{5mm}
  \begin{answer}
    \texttt{git remote -v}
  \end{answer}
\end{question}
\end{block}

\begin{block}[questions=1]

You are working on a team project.  You have forked the main project
repository, and cloned your fork.  This automatically gave you a remote called
\texttt{origin} that points to your forked repository.  You added a remote
called \texttt{upstream} that points to the main project repository.

Overnight, your eager team members have merged some work to the main
repository, therefore updating the \texttt{master} branch of the main
repository.  Today you want to do some work on a new branch that starts off at
the new position of \texttt{master} on the main repository.

\begin{question}

    What command(s) would you type to start working in a new branch named
    \texttt{more-work} that starts at the new position of \texttt{master} in
    the main repository?

    \vspace{80mm}
    \begin{answer}
        \texttt{
            git fetch upstream \\
            git branch more-work --no-track upstream/master \\
            git checkout master
        }
    \end{answer}

\end{question}

\begin{question}

    You have done some commits on your new branch \texttt{more-work}.  What
    command(s) would you type to replicate this your new branch to your own
    fork on Github?

    \vspace{80mm}
    \begin{answer}
        \texttt{
            git push origin more-work --set-upstream
        }
    \end{answer}

\end{question}

\end{block}

\end{shortanswer}

\end{document}
